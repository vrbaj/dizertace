\chapter{Algoritmus Extreme Seeking Entropy}
V této kapitole je představen algoritmus pro detekci novosti nazvaný Extreme Seeking Entropy (ESE), který tvoří hlavní výsledek předkládané dizertační práce. Návržený algoritmus, vychází z předpokladu, že novost v datech se projeví neobvykle velkými přírůstky vah adaptivního filtru, který danou řadu dat modeluje. Cílem je tedy vyhodnotit neobvykle velké přírůstky vah adaptivního systém. Je tedy nejprve nutné určit nějaký práh $z$, podle kterého můžeme přírůstky adaptivních vah filtru rozdělit do dvou množin. Množinu, která bude obsahovat obsahovat přírůstky menší než je zvolený práh $z$ označíme $L$. Přírůstek který je větší nebo roven hodnotě prahu $z$ označíme $H$. Pozn. Volba hodnoty prahu přímo souvisí s volbou metody POT (viz kapitola \ref{chap:gpd}). Uvažujme, že obě množiny existují pro každou adaptivní váhu, potom pro $i$-tou adaptivní váhu zvolíme práh $z_i$ tak, že velikosti přírůstku této váhy náleží do jedné ze dvou množin, tak, že:
\begin{equation}
\forall \abs{\Delta w_i} < z_i \in L_i
\end{equation}
\begin{equation}
\forall \abs{\Delta w_i} \geq z_i \in H_i
\end{equation}
Vzhledem k výše zmíněnému předpokladu o velikosti změn vah adaptivního filtru a novosti v datech, uvažujme, že přírůstky náležející množině $L_i$ pravděpodobně neobsahují informaci o novosti během adaptace a proto nebudou vyhodnocovány. Množina $H_i$ by měla obsahovat přírůstky

\par Uvažujme, že máme dokonale nastavený adaptivní filtr, jehož chyba predikce $e$ je nulová pro všechny vstupní hodnoty. Potom přírůstky vah tohoto filtru budou nulové. V případě, že dojde k nějaké změně v generátoru dat pro tento filtr, začne se filtr opět adaptovat což vyústí v nenulové změny adaptivních vah, které budou reflektovat novost způsobenou změnou vlastností daného generátoru.

\begin{equation}
ESE(\abs{\Delta \textbf{w}(k)})=-\log \prod_{i=1}^n(1-f_{cdf_i}(\abs{\Delta w_i(k)}))
\end{equation}
kde
\begin{equation}
f_{cdf_i}(\abs{\Delta w_i(k)})=
\end{equation}