\chapter{Závěr}
Předložená dizertační práce je věnována použití adaptivních systémů při analýze dat. Z chronologického pohledu je prvním výstupem během zpracování této práce případová studie algoritmu Learning Entropy pro detekci změn stavu bioprocesu s využitím adaptivního fuzzy filtru \cite{artep}, více viz příloha \ref{chap:LE_fuzzy}. Dále byla úspěšně vyzkoušena detekce fázové změny v krystalografických datech pomocí adaptivního filtru a zobecněného rozdělení extrémních hodnot \cite{asr}. Zobecněné rozdělení extrémních hodnot bylo také využito v případové studii skokové změny parametrů generátoru signálu \cite{appel1}. 
\par 
Za zásadní výsledek lze považovat nový originální algoritmus pro detekci novosti, který vyhodnocuje přírůstky adaptivních vah filtru, Extreme Seeking Entropy \cite{ese_mdpi} (viz kapitola \ref{chap:ese}). Tento algoritmus byl otestován v následujících případových studiích (viz kapitola \ref{chap:vysledky}): detekce pertubace v chaotické časové řadě získané řešením Mackey-Glassovy rovnice, detekce změny rozptylu šumu v náhodném datovém toku, detekce skokové změny parametrů generátoru signálu, detekce náhlé absence šumu, detekce změny trendu a při detekci epilepsie v myším EEG. Pro detekci skokové změny trendu a skokové změny parametrů signálu byla vyhodnocena úspěšnost této detekce a výsledky porovnány s výsledky algoritmů Learning Entropy a Error and Learning Based Novelty Detection, přičemž v obou případech byla úspěšnost detekce algoritmu ESE vyšší pro téměř všechny vyhodnocované hodnoty SNR. Pro hodnotu $SNR>34$ $dB$ dosáhl algoritmus při detekci skokové změny parametrů generátoru signálu $100\%$ úspěšnost. Při detekci změny trendu měl algoritmus ESE pro hodnoty $SNR>8$ $dB$ větší úspěšnost detekce než srovnávané algoritmy LE a ELBND. Výše uvedené výsledky byly publikovány v \cite{ese_mdpi}.
\par Pro možné použití v aplikacích detekce v reálném čase byla experimentálně zjišťována výpočetní časová náročnost různých metod odhadů parametrů zobecněného Paretova rozdělení v případě použití algoritmu ESE při detekci skokové změny parametrů \cite{appel2}. Výsledkem je porovnání 3 různých metod odhadu parametrů. Limitujícím faktorem použití ESE v reálném čase je v zásadě počet adaptivních parametrů filtru, které je potřeba vyhodnocovat a samozřejmě rychlost vzorkování monitorovaného signálu.
\par Pro odhad úspěšnosti detekce novosti pomocí algoritmu ESE byla také vyhodnocena ROC křivka v případě detekce změny trendu signálu s různými poměry SNR a byly určeny příslušné plochy pod těmito ROC křivkami \cite{appel3}. Dosažené výsledky byly opět porovnány s algoritmy LE a ELBND a bylo ověřeno, že pro hodnoty $SNR \leq 30$ $dB$ dosahuje algoritmus ESE lepších výsledků. Pro vyšší hodnoty $SNR$ pak byly výsledky ESE srovnatelné s výsledky LE.  Cílem této studie bylo zjistit jak dobře dokáže algoritmus ESE separovat nová data v závislosti na volbě prahu, který rozhoduje o tom zda data obsahují novost či nikoliv. Pro představu ještě uveďme, že např. pro hodnotu $SNR=16.2$ $dB$ bylo dosaženo úspěšnosti detekce $90.42\%$.
\par
Stanovené cíle dizertační práce (viz \ref{chap:cile}) tak lze, na základě výše uvedených výsledků, považovat za splněné.

\section{Možné směry budoucího výzkumu}
V budoucnu se nabízí rozvíjet téma využití adaptivních systémů ve zpracování dat několika směry. Potenciální využití vyhodnocení změn vah adaptivních systémů lze využít v optimalizaci velikosti datasetů v oblasti hlubokého učení, což by mohlo výrazně snížit časovou náročnost učení hlubokých sítí. Za účelem snížení výpočetního času algoritmu ESE je potřeba vyzkoušet další metody odhadu parametrů zobecněného Paretova rozdělení a vyzkoušet adaptivní metody volby velikosti prahu pro metodu Peak-over-threshold. Zajímavým tématem je také vliv šumu a jeho typu na velikost přírůstků adaptivních vah filtrů a jejich pravděpodobnostní rozdělení. V neposlední řadě se nabízí otázka, jak ovlivní typ kriteriální funkce pro optimalizaci adaptivního filtru výsledky algoritmu ESE a je-li možné různé kriteriální funkce využívat k detekci novosti, případně pomocí nich typ novosti klasifikovat. 