\chapter*{Úvod}

Tato disertační práce je věnována problematice využití adaptivních systémů při analýze dat. Vzhledem k exponenciálnímu celosvětovému nárustu dat a ke zvyšování jejich variability roste i potřeba tato data analyzovat, kategorizovat a vytěžovat (TODO: nějaká citace k nárůstu dat). Analýzou dat rozumíme proces, kdy z nezpracovaných naměřených dat získáme nějakou interpretovatelnou informaci, s kterou pak lze dál pracovat. Jedna z možných důležitých interpretací nově získaných dat je, zda-li se nově získaná data nějakým zásadním způsobem odlišují od předchozích dat. Této problematice se věnuje obor detekce novosti, neboli anomálií, který spadá do oblasti vytěžování dat a strojového učení. Úspěšná detekce novosti pak může být využita k vícero účelům. Například k diagnostice sledovaného procesu, ke změně struktury nebo parametrů adaptivního modelu za účelem zlepšení predikce, z konkrétních aplikací pak k odhalení neoprávněného vniknutí do sítě nebo zneužití dat, v lékařství se detekce novosti používá k diagnostickým účelům, z průmyslových aplikací pak k detekci poruchy a monitoringu stavu strojů, senzorů, ev zpracování textových dat k detekci nových témat atd. Spektrum využití je velice široké.
\par
V oblasti detekce novosti byla v posledních desetiletích intenzivního vývoje navrhnuta celá řada algoritmů. Vzhledem k rostoucímu výpočetnímu výkonu a rozmanitosti analyzovaných dat rostla i potřeba nových algoritmů. Nové algoritmy typicky předčili ostatní algoritmy v rámci jedné aplikace, respektive v rámci jednoho typu dat.  Doposud se však nepodařilo vytvořit algoritmus, který by ve všech, nebo alespoň ve významné části, oblastech použití předčil již publikované algoritmy. I proto vznikají v oblasti detekce novosti neustále nové přístupy, které navíc umožňují analyzovat nové typy dat.

Předkládáná disertační práce je členěna do pěti kapitol. První kapitola je věnována přehledu různých metod detekce novosti a obsahuje i oblasti jejich využití. Druhá kapitola obsahuje přehled adaptivních filtrů a metod, které byly v rámci práce použity. Třetí kapitola je věnována zobecněnému Paretovu rozdělení, které bylo použito v navrženém algoritmu detekce novosti. Čtvrtá kapitola obsahuje popis nově navrženého algoritmu nazvaném Extreme Seeking Entropy a představuje možnosti jeho použití v různých případech. Dále jsou zde výsledky tohoto algoritmu v porovnání s dalšími vybranými metodami adaptivní detekce novosti. 